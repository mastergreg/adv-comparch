%{{{ preamble
\documentclass[a4paper,12pt]{article}
\usepackage{anysize}
\marginsize{2cm}{2cm}{1cm}{1cm}
%0.7\textwidth 6.0in \textheight = 664pt
\usepackage{xltxtra}
\usepackage{xunicode}
\usepackage{graphicx}
\usepackage{color}
\usepackage{xgreek}
\usepackage{fancyvrb}
\usepackage{minted}
\usepackage{listings}
\usepackage{enumitem} 
\usepackage{framed} 
\usepackage{relsize}
\usepackage{float} 
\usepackage{pstricks}
\usepackage{pst-node}
\usepackage{pst-blur}
\usepackage{multirow}
\setmainfont[Mapping=tex-text]{FreeSerif}
%}}}
\begin{document}

\def\thepart{\Alph{part}}
%\def\thesection {\alpha{section}. }
%\def\thesubsection {\section.\arabic{subsection}. }

\include{title/title}

\part{}

\section{Εισαγωγή}
Στην άσκηση αυτή προσομοιώνουμε με χρήση του Simics τον πολλαπλασιασμό δύο
τετραγωνικών πινάκων Α και Β χρησιμοποιώντας διάφορες τεχνικές
βελτιστοποίησης. 


%{{{ Προσομοίωση
\section{Περιβάλλον Προσομοίωσης}
Για το εκτελέσιμο χρησιμοποιούμε γλώσσα C και gcc, χωρίς
optimization flags.

\begin{minted}{bash}
    target$ gcc -O1 -o executable partA.c
\end{minted}
%}}}

%{{{ Προσομοίωση

\subsection{Προσομοίωση}
Ο κώδικας που μας δόθηκε ήταν ο ακόλουθος:

\inputminted[linenos,fontsize=\scriptsize,frame=leftline]{c}{files/partA1-default-partA.c}

%}}}

\pagebreak

%{{{ Ιεραρχία μνήμης και μοντέλο απόδοσης

\subsection{Ιεραρχία μνήμης και μοντέλο απόδοσης}

Χρησιμοποιήσαμε την ιεραρχία μνήμης όπως μας δίνεται στο Παράρτημα. Αυτή έχει
δύο επίπεδα κρυφής μνήμης L1 (2 way set associative 64 bytes $\times$ 512 lines
write-through LRU policy) και
L2 (4 way set associative 128 bytes $\times$ 1024 lines write-back LRU policy).

\begin{table}[H]
    \centering
    \begin{tabular}{| c | c | c | c | c |}
        \hline
 & assoc &   line size &   lines & size\\
        \hline
        \hline
        L1 instruction cache & 2 &   64 &   512 & 32768 = 32 KB\\
        \hline
        L1 data cache & 2 &   64 &   512 & 32768 = 32 KB\\
        \hline
        L2 cache & 4 &   128 &   1024 & 131072 = 128 KB\\
        \hline
    \end{tabular}
    \caption{Cache Hierarchy}
    \label{fig:T2}
\end{table}

Στο μοντέλο που προσομοιώνουμε οι κύκλοι υπολογίζονται χωρίς penalty από το
Simics. Για να υπολογίσουμε τους κύκλους θεωρούμε πως πρόσβαση στην L1 cache
γίνεται σε 1 κύκλο, στην L2 σε 20 κύκλους και στη μνήμη σε 300 κύκλους οπότε
έχουμε τον τύπο:

\begin{equation}
    Cycles = Inst + L1_{Accesses} * L1_{Time} + L2_{Accesses} * L2_{Time} +
    Mem_{Accesses} * Mem_{Time}
\end{equation} 

\subsection{Αρχική έκδοση}

Προσομοιώσαμε τον αρχικό κώδικα, όπως φαίνεται παραπάνω και πήραμε τα ακόλουθα
αποτελέσματα.

\begin{table}[H]
    \centering
    \begin{tabular}{| c | c |}
        \hline
        Cycles & 1523400216 \\
        \hline
        L1 miss ratio & 0.029 \\
        \hline
        L2 miss ratio & 0.015 \\
        \hline
    \end{tabular}
    \caption{Μετρικές πρώτης εκτέλεσης}
    \label{fig:T3a}
\end{table}

%}}}

\section{Τεχνικές Βελτιστοποίησης}
%{{{ Loop Interchange

\subsection{Loop Interchange}

Για την προσομοίωση κάναμε 6 εκτελέσεις για κάθε διάταξη των i,j,k όπως
φαίνεται στον πίνακα \ref{fig:T1}.

\begin{table}[H]
    \centering
    \begin{tabular}{| c | c |}
        \hline
        Σειρά προσομοίωσης & Διάταξη \\
        \hline
        \hline
        1 &  i,j,k \\
        \hline
        2 &  i,k,j \\
        \hline
        3 &  j,i,k \\
        \hline
        4 &  j,k,i \\
        \hline
        5 &  k,i,j \\
        \hline
        6 &  k,j,i \\
        \hline
    \end{tabular}
    \caption{Δυνατές Διατάξεις}
    \label{fig:T1}
\end{table}


Τα αποτελέσματα φαίνονται στα ακόλουθα σχήματα.

\begin{figure}[H]
	\centering
    \includegraphics[width=0.8\textwidth]{files/partA1-simics-outputs-cycles.png}
	\caption{Cycles}
	\label{fig:A1}
\end{figure}

\begin{figure}[H]
	\centering
    \includegraphics[width=0.8\textwidth]{files/partA1-simics-outputs-l1_miss.png}
    \caption{L1 Miss Rate}
	\label{fig:A2}
\end{figure}

\begin{figure}[H]
	\centering
    \includegraphics[width=0.8\textwidth]{files/partA1-simics-outputs-l2_miss.png}
    \caption{L2 Miss Rate}
	\label{fig:A3}
\end{figure}

\begin{table}[H]
    \centering
    \begin{tabular}{| c | c | c | c |}
        \hline
        Cycles & L1 Miss rate & L2 Miss rate & Speedup  \\
        \hline
        \hline
        1523400216 & 0.0285 & 0.0150 & 1.0 \\
        1519770751 & 0.0046 & 0.0152 & 1.00 \\
        1145990549 & 0.0303 & 0.0281 & 1.32 \\
        2752007165 & 0.0835 & 0.0675 & 0.55 \\
        1905041698 & 0.0045 & 0.0115 & 0.79 \\
        2659818958 & 0.0833 & 0.0619 & 0.57 \\
        \hline
    \end{tabular}
    \caption{Μετρικές όλων των εκτελέσεων}
    \label{fig:T3b}
\end{table}


Ανάλογα με τη διάταξη των loops οι πίνακες διασχίζονται κατά γραμμές και κατά
στήλες σε διάφορους συνδυασμούς. Μεγαλύτερο speedup παρατηρούμε στη διάταξη
j,i,k.

%}}}

\pagebreak

%{{{ Cache Blocking

\subsection{Cache Blocking}

Η L1 cache έχει μέγεθος 32KB, και line size 64 bytes. Κάνουμε πράξεις μεταξύ
αριθμών κινητής υποδιαστολής σε 32bit σύστημα συνεπώς κάθε ένας έχει μέγεθος
4bytes. Κάθε cache line χωράει 16 αριθμούς συνεπώς για να εκμεταλλευτούμε
καλύτερα την τοπικότητα των αναφορών θα εκτελούμε κάθε loop σε ομάδες. Έτσι θα 
χρησιμοποιούμε δεδομένα που έχουν ήδη έρθει στην cache.

\inputminted[linenos,fontsize=\scriptsize,frame=leftline]{c}{files/partA2_scan-v3-partA.c}


\begin{table}[H]
    \centering
    \begin{tabular}{| c | c | c | c |}
        \hline
        Cycles & L1 Miss rate & L2 Miss rate & Speedup  \\
        \hline
        \hline
        1229008147 & 0.00200 & 0.00436 & 1.0 \\
        1116424502 & 0.00222 & 0.00355 & 1.1008 \\
        1090971021 & 0.00276 & 0.00363 & 1.1265 \\
        1077600084 & 0.00364 & 0.00337 & 1.1405 \\
        1074879790 & 0.00406 & 0.00347 & 1.1433 \\
        1077724197 & 0.00460 & 0.00443 & 1.1403 \\
        1076847331 & 0.00506 & 0.00464 & 1.1413 \\
        1092092535 & 0.00608 & 0.00734 & 1.1253 \\
        1080181105 & 0.00554 & 0.00586 & 1.1377 \\
        1082269217 & 0.00561 & 0.00616 & 1.1355\\
        1085119465 & 0.00598 & 0.00724 & 1.1326 \\
        1089671800 & 0.00604 & 0.00797 & 1.1278 \\
        1093319686 & 0.00628 & 0.00837 & 1.1241 \\
        1106283098 & 0.00649 & 0.01040 & 1.1109 \\
        1134938000 & 0.00634 & 0.01550 & 1.0828 \\
        1150045093 & 0.00692 & 0.01869 & 1.0686 \\
        \hline
    \end{tabular}
    \caption{Διαφορά των εκτελέσεων}
    \label{fig:T4}
\end{table}

\begin{figure}[H]
	\centering
    \includegraphics[width=0.8\textwidth]{files/partA2_scan-simics-outputs-cycles_hist.png}
	\caption{Cycles}
	\label{fig:B1}
\end{figure}

\begin{figure}[H]
	\centering
    \includegraphics[width=0.8\textwidth]{files/partA2_scan-simics-outputs-l1_miss_hist.png}
    \caption{L1 Miss Rate}
	\label{fig:B2}
\end{figure}

\begin{figure}[H]
	\centering
    \includegraphics[width=0.8\textwidth]{files/partA2_scan-simics-outputs-l2_miss_hist.png}
    \caption{L2 Miss Rate}
	\label{fig:B3}
\end{figure}


Η απλοική εκτέλεση διαρκεί 1523400216 cycles. H βέλτιστη cache blocked
1074879790. Το speedup που υπολογίζουμε από αυτή τη διαφορά είναι 1.41.


Παρατηρούμε πως η μεταβολή του block size στο πρόγραμμά μας, αν και αυξάνει το πλήθος των
εντολών του προγράμματος, αυξάνει την ταχύτητά του καθώς εκμεταλλεύεται
καλύτερο την τοπικότητα των αναφορών. Ωστόσο για ακραίες τιμές η επιτάχυνση
δεν είναι πολύ σημαντική. Είναι μια τεχνική που μπορεί να εφαρμοστεί
ταυτόχρονα με την αναδιάρθρωση των βρόχων αρκεί το pattern να μην είναι τυχαίο
αλλά η πρόσβαση στα δεδομένα να μπορεί να μετασχηματιστεί σε αντίστοιχη μορφή.

%}}}

\setcounter{section}{0}
\part{}
\section{Πρωτόκολλο MESI}

Έχουμε ένα πολυεπεξεργαστικό σύστημα δύο επεξεργαστών που χρησιμοποιεί το
πρωτόκολλο MESI. Κάθε επεξεργαστής έχει cache με τα ακόλουθα χαρακτηριστικά.

\begin{table}[H]
    \centering
    \begin{tabular}{r l}
        Cache Size : & 4KB \\
        Associativity : & 2-way set write-back \\
        Set Size : & 2KB \\
        Block Size : & 16 bytes $\rightarrow$ 4 bits block offset\\
        Blocks/way : & 2KB / way $\rightarrow$ $2^{11}/2^4=2^7=128$ blocks / way \\
        Tag : & 5 bits \\
        Policy : & LRU \\
        Address Size : & 16 bytes \\
    \end{tabular}
    \caption{Cache characteristics}
    \label{fig:T5}
\end{table}

Συνεπώς μπορούμε να βρούμε την αντιστοιχία των διευθύνσεων στην cache.

\begin{table}[H]
    \centering
    \begin{tabular}{c | c c c}
        Addresses & Tag & Index & Block Offset \\
        \hline
        073C & 00000 & 1110011 & 1100 \\
        0734 & 00000 & 1110011 & 0100 \\
        0738 & 00000 & 1110011 & 1000 \\
        0730 & 00000 & 1110011 & 0000 \\
        1F34 & 00011 & 1110011 & 0100 \\
        1F3C & 00011 & 1110011 & 1100 \\
        2730 & 00100 & 1110011 & 0000 \\
        273C & 00100 & 1110011 & 1100 \\
    \end{tabular}
    \caption{Memory $\mapsto$ Cache}
    \label{fig:T6}
\end{table}


\begin{table}[H]
    \centering
    \begin{tabular}{| c | c c |}
            \hline
 & P0 & P1                    \\
            \hline
        1 & read 073C &                       \\
            \hline
        2 & & read 0734             \\
            \hline
        3 & & write‘1111’ -> 0734   \\
            \hline
        4 & read 0738 &                       \\
            \hline
        5 & write‘2222’ -> 0730 &                       \\
            \hline
        6 & & write‘3333’-> 1F34    \\
            \hline
        7 & write‘4444’ -> 1F3C &                       \\
            \hline
        8 & read 073C &                       \\
            \hline
        9 & & write‘5555’-> 2730    \\
            \hline
        10 & read 273C &                       \\
            \hline
        11 & & write‘6666’-> 273C    \\
            \hline
    \end{tabular}
    \caption{Σειρά προσπελάσεων στη μνήμη}
    \label{fig:T7}
\end{table}


\begin{table}[H]
    \centering
    \caption{P0: read 073C}
    \begin{tabular}{| c || c | c | c | c || c |}
        \hline
        \multicolumn{6}{| c |}{Memory}  \\
        \hline
        \hline
        Address & 0 & 4 & 8 & C & Valid \\
        \hline
        073X & 0000 & 0000 & 0000 & 0000 & V \\
        1F3X & 0000 & 0000 & 0000 & 0000 & V \\
        273X & 0000 & 0000 & 0000 & 0000 & V \\
        \hline
    \end{tabular}
\end{table}


\begin{table}[H]
    \centering
    \begin{tabular}{| c || c | c | c | c | c | c | c |}
        \hline
        Processor & Tag & Set/line & MESI & \multicolumn{4}{| c |}{Data} \\
        \hline
        \hline
        \multirow{2}{*}{Processor 0} 
 & 0 & 115/0 & E & 0000 & 0000 & 0000 & 0000 \\
        \cline{2-8}
 & & & & & & & \\
        \hline
        \hline
        \multirow{2}{*}{Processor 1} & & & & & & & \\
        \cline{2-8}
 & & & & & & & \\
        \hline
    \end{tabular}
\end{table}


\begin{table}[H]
    \centering
    \caption{P1: read 0734}
    \begin{tabular}{| c || c | c | c | c || c |}
        \hline
        \multicolumn{6}{| c |}{Memory}  \\
        \hline
        \hline
        Address & 0 & 4 & 8 & C & Valid \\
        \hline
        073X & 0000 & 0000 & 0000 & 0000 & V \\
        1F3X & 0000 & 0000 & 0000 & 0000 & V \\
        273X & 0000 & 0000 & 0000 & 0000 & V \\
        \hline
    \end{tabular}
\end{table}


\begin{table}[H]
    \centering
    \begin{tabular}{| c || c | c | c | c | c | c | c |}
        \hline
        Processor & Tag & Set/line & MESI & \multicolumn{4}{| c |}{Data} \\
        \hline
        \hline
        \multirow{2}{*}{Processor 0} 
 & 0 & 115/0 & S & 0000 & 0000 & 0000 & 0000 \\
        \cline{2-8}
 & & & & & & & \\
        \hline
        \hline
        \multirow{2}{*}{Processor 1}
 & 0 & 115/0 & S & 0000 & 0000 & 0000 & 0000 \\
        \cline{2-8}
 & & & & & & & \\
        \hline
    \end{tabular}
\end{table}


\begin{table}[H]
    \centering
    \caption{P1: write '1111' to 0734}
    \begin{tabular}{| c || c | c | c | c || c |}
        \hline
        \multicolumn{6}{| c |}{Memory}  \\
        \hline
        \hline
        Address & 0 & 4 & 8 & C & Valid \\
        \hline
        073X & 0000 & 0000 & 0000 & 0000 & I \\
        1F3X & 0000 & 0000 & 0000 & 0000 & V \\
        273X & 0000 & 0000 & 0000 & 0000 & V \\
        \hline
    \end{tabular}
\end{table}


\begin{table}[H]
    \centering
    \begin{tabular}{| c || c | c | c | c | c | c | c |}
        \hline
        Processor & Tag & Set/line & MESI & \multicolumn{4}{| c |}{Data} \\
        \hline
        \hline
        \multirow{2}{*}{Processor 0} 
 & 0 & 115/0 & I & - & - & - & - \\
        \cline{2-8}
 & & & & & & & \\
        \hline
        \hline
        \multirow{2}{*}{Processor 1}
 & 0 & 115/0 & M & 0000 & 0000 & 0000 & 1111 \\
        \cline{2-8}
 & & & & & & & \\
        \hline
    \end{tabular}
\end{table}


\begin{table}[H]
    \centering
    \caption{P0: read 0738}
    \begin{tabular}{| c || c | c | c | c || c |}
        \hline
        \multicolumn{6}{| c |}{Memory}  \\
        \hline
        \hline
        Address & 0 & 4 & 8 & C & Valid \\
        \hline
        073X & 0000 & 0000 & 0000 & 1111 & V \\
        1F3X & 0000 & 0000 & 0000 & 0000 & V \\
        273X & 0000 & 0000 & 0000 & 0000 & V \\
        \hline
    \end{tabular}
\end{table}


\begin{table}[H]
    \centering
    \begin{tabular}{| c || c | c | c | c | c | c | c |}
        \hline
        Processor & Tag & Set/line & MESI & \multicolumn{4}{| c |}{Data} \\
        \hline
        \hline
        \multirow{2}{*}{Processor 0} 
 & 0 & 115/0 & S & 0000 & 0000 & 0000 & 1111 \\
        \cline{2-8}
 & & & & & & & \\
        \hline
        \hline
        \multirow{2}{*}{Processor 1}
 & 0 & 115/0 & S & 0000 & 0000 & 0000 & 1111 \\
        \cline{2-8}
 & & & & & & & \\
        \hline
    \end{tabular}
\end{table}


\begin{table}[H]
    \centering
    \caption{P0: write '2222' to 0730}
    \begin{tabular}{| c || c | c | c | c || c |}
        \hline
        \multicolumn{6}{| c |}{Memory}  \\
        \hline
        \hline
        Address & 0 & 4 & 8 & C & Valid \\
        \hline
        073X & 0000 & 0000 & 0000 & 1111 & I \\
        1F3X & 0000 & 0000 & 0000 & 0000 & V \\
        273X & 0000 & 0000 & 0000 & 0000 & V \\
        \hline
    \end{tabular}
\end{table}


\begin{table}[H]
    \centering
    \begin{tabular}{| c || c | c | c | c | c | c | c |}
        \hline
        Processor & Tag & Set/line & MESI & \multicolumn{4}{| c |}{Data} \\
        \hline
        \hline
        \multirow{2}{*}{Processor 0} 
 & 0 & 115/0 & M & 2222 & 0000 & 0000 & 1111 \\
        \cline{2-8}
 & & & & & & & \\
        \hline
        \hline
        \multirow{2}{*}{Processor 1}
 & 0 & 115/0 & I & - & - & - & - \\
        \cline{2-8}
 & & & & & & & \\
        \hline
    \end{tabular}
\end{table}


\begin{table}[H]
    \centering
    \caption{P1: write '3333' to 1F34}
    \begin{tabular}{| c || c | c | c | c || c |}
        \hline
        \multicolumn{6}{| c |}{Memory}  \\
        \hline
        \hline
        Address & 0 & 4 & 8 & C & Valid \\
        \hline
        073X & 0000 & 0000 & 0000 & 1111 & I \\
        1F3X & 0000 & 0000 & 0000 & 0000 & I \\
        273X & 0000 & 0000 & 0000 & 0000 & V \\
        \hline
    \end{tabular}
\end{table}


\begin{table}[H]
    \centering
    \begin{tabular}{| c || c | c | c | c | c | c | c |}
        \hline
        Processor & Tag & Set/line & MESI & \multicolumn{4}{| c |}{Data} \\
        \hline
        \hline
        \multirow{2}{*}{Processor 0} 
 & 0 & 115/0 & M & 2222 & 0000 & 0000 & 1111 \\
        \cline{2-8}
 & & & & & & & \\
        \hline
        \hline
        \multirow{2}{*}{Processor 1}
 & 0 & 115/0 & I & - & - & - & - \\
        \cline{2-8}
 & 3 & 115/1 & M & 0000 & 3333 & 0000 & 1111 \\
        \hline
    \end{tabular}
\end{table}


\begin{table}[H]
    \centering
    \caption{P0: write '4444' to 1F3C}
    \begin{tabular}{| c || c | c | c | c || c |}
        \hline
        \multicolumn{6}{| c |}{Memory}  \\
        \hline
        \hline
        Address & 0 & 4 & 8 & C & Valid \\
        \hline
        073X & 0000 & 0000 & 0000 & 1111 & I \\
        1F3X & 0000 & 3333 & 0000 & 0000 & I \\
        273X & 0000 & 0000 & 0000 & 0000 & V \\
        \hline
    \end{tabular}
\end{table}


\begin{table}[H]
    \centering
    \begin{tabular}{| c || c | c | c | c | c | c | c |}
        \hline
        Processor & Tag & Set/line & MESI & \multicolumn{4}{| c |}{Data} \\
        \hline
        \hline
        \multirow{2}{*}{Processor 0} 
 & 0 & 115/0 & M & 2222 & 0000 & 0000 & 1111 \\
        \cline{2-8}
 & 3 & 115/1 & M & 0000 & 3333 & 0000 & 4444 \\
        \hline
        \hline
        \multirow{2}{*}{Processor 1}
 & 0 & 115/0 & I & - & - & - & - \\
        \cline{2-8}
 & 3 & 115/1 & I & - & - & - & - \\
        \hline
    \end{tabular}
\end{table}


\begin{table}[H]
    \centering
    \caption{P0: read 073C}
    \begin{tabular}{| c || c | c | c | c || c |}
        \hline
        \multicolumn{6}{| c |}{Memory}  \\
        \hline
        \hline
        Address & 0 & 4 & 8 & C & Valid \\
        \hline
        073X & 0000 & 0000 & 0000 & 1111 & I \\
        1F3X & 0000 & 3333 & 0000 & 0000 & I \\
        273X & 0000 & 0000 & 0000 & 0000 & V \\
        \hline
    \end{tabular}
\end{table}


\begin{table}[H]
    \centering
    \begin{tabular}{| c || c | c | c | c | c | c | c |}
        \hline
        Processor & Tag & Set/line & MESI & \multicolumn{4}{| c |}{Data} \\
        \hline
        \hline
        \multirow{2}{*}{Processor 0} 
 & 0 & 115/0 & M & 2222 & 0000 & 0000 & 1111 \\
        \cline{2-8}
 & 3 & 115/1 & M & 0000 & 3333 & 0000 & 4444 \\
        \hline
        \hline
        \multirow{2}{*}{Processor 1}
 & 0 & 115/0 & I & - & - & - & - \\
        \cline{2-8}
 & 3 & 115/1 & I & - & - & - & - \\
        \hline
    \end{tabular}
\end{table}


\begin{table}[H]
    \centering
    \caption{P1: write '5555' to 2730}
    \begin{tabular}{| c || c | c | c | c || c |}
        \hline
        \multicolumn{6}{| c |}{Memory}  \\
        \hline
        \hline
        Address & 0 & 4 & 8 & C & Valid \\
        \hline
        073X & 0000 & 0000 & 0000 & 1111 & I \\
        1F3X & 0000 & 3333 & 0000 & 0000 & I \\
        273X & 0000 & 0000 & 0000 & 0000 & I \\
        \hline
    \end{tabular}
\end{table}


\begin{table}[H]
    \centering
    \begin{tabular}{| c || c | c | c | c | c | c | c |}
        \hline
        Processor & Tag & Set/line & MESI & \multicolumn{4}{| c |}{Data} \\
        \hline
        \hline
        \multirow{2}{*}{Processor 0} 
 & 0 & 115/0 & M & 2222 & 0000 & 0000 & 1111 \\
        \cline{2-8}
 & 3 & 115/1 & M & 0000 & 3333 & 0000 & 4444 \\
        \hline
        \hline
        \multirow{2}{*}{Processor 1}
 & 4 & 115/0 & M & 5555 & 0000 & 0000 & 0000 \\
        \cline{2-8}
 & 3 & 115/1 & I & - & - & - & - \\
        \hline
    \end{tabular}
\end{table}


\begin{table}[H]
    \centering
    \caption{P0: read 273C}
    \begin{tabular}{| c || c | c | c | c || c |}
        \hline
        \multicolumn{6}{| c |}{Memory}  \\
        \hline
        \hline
        Address & 0 & 4 & 8 & C & Valid \\
        \hline
        073X & 0000 & 0000 & 0000 & 1111 & I \\
        1F3X & 0000 & 3333 & 0000 & 4444 & V \\
        273X & 5555 & 0000 & 0000 & 0000 & V \\
        \hline
    \end{tabular}
\end{table}


\begin{table}[H]
    \centering
    \begin{tabular}{| c || c | c | c | c | c | c | c |}
        \hline
        Processor & Tag & Set/line & MESI & \multicolumn{4}{| c |}{Data} \\
        \hline
        \hline
        \multirow{2}{*}{Processor 0} 
 & 0 & 115/0 & M & 2222 & 0000 & 0000 & 1111 \\
        \cline{2-8}
 & 4 & 115/1 & S & 5555 & 0000 & 0000 & 0000 \\
        \hline
        \hline
        \multirow{2}{*}{Processor 1}
 & 4 & 115/0 & S & 5555 & 0000 & 0000 & 0000 \\
        \cline{2-8}
 & 3 & 115/1 & I & - & - & - & - \\
        \hline
    \end{tabular}
\end{table}


\begin{table}[H]
    \centering
    \caption{P0: write '6666' to 273C}
    \begin{tabular}{| c || c | c | c | c || c |}
        \hline
        \multicolumn{6}{| c |}{Memory}  \\
        \hline
        \hline
        Address & 0 & 4 & 8 & C & Valid \\
        \hline
        073X & 0000 & 0000 & 0000 & 1111 & I \\
        1F3X & 0000 & 3333 & 0000 & 4444 & V \\
        273X & 5555 & 0000 & 0000 & 0000 & I \\
        \hline
    \end{tabular}
\end{table}


\begin{table}[H]
    \centering
    \begin{tabular}{| c || c | c | c | c | c | c | c |}
        \hline
        Processor & Tag & Set/line & MESI & \multicolumn{4}{| c |}{Data} \\
        \hline
        \hline
        \multirow{2}{*}{Processor 0} 
 & 0 & 115/0 & M & 2222 & 0000 & 0000 & 1111 \\
        \cline{2-8}
 & 4 & 115/1 & M & 5555 & 0000 & 0000 & 6666 \\
        \hline
        \hline
        \multirow{2}{*}{Processor 1}
 & 4 & 115/0 & I & - & - & - & - \\
        \cline{2-8}
 & 3 & 115/1 & I & - & - & - & - \\
        \hline
    \end{tabular}
\end{table}

\pagebreak
\section{Memory Consistency}
Οι δυνατοί συνδιασμοί των τελικών τιμών φαίνονται στους παρακάτω πίνακες.
\begin{table}[H]
    \centering
    \begin{tabular}{|c|c| |c|c| |c|c| |c|c|}
        \hline
        P0 &     P1 &       P0 &     P1 &       P0 &     P1 &      P0 &     P1       \\
        \hline
        flag=1 & &     Y=1 & &     X=2 & &    X=2 &         \\
               &    r3=0 &    flag=1 & &    flag=1 & &    Y=1 &                  \\
               &    r4=0 & &    r3=0 & &    r3=2 &   flag=1 &                  \\
               &   flag=0 & &    r4=1 & &    r4=0 & &    r3=2      \\
        X=2 & & &   flag=0 & &   flag=0 & &    r4=1      \\
        Y=1 & &    X=2 & &     Y=1 & & &   flag=0     \\
        r1=2 & &    r1=2 & &     r1=2 & &   r1=2 &               \\
        r2=0 & &    r2=0 & &     r2=0 & &   r2=0 &               \\
             &     Z=4 & &     Z=4 & &     Z=4 & &     Z=4      \\
        \hline
        \multicolumn{2}{|c||}{(r1,r2,r3,r4)=(2,0,0,0)} & \multicolumn{2}{|c||}{(r1,r2,r3,r4)=(2,0,0,1)} &
        \multicolumn{2}{|c||}{(r1,r2,r3,r4)=(2,0,2,0)} &   \multicolumn{2}{|c|}{(r1,r2,r3,r4)=(2,0,2,1)} \\
        \hline
        \hline
        P0 & P1 & P0 &        P1 &         P0 &  P1 &  P0 &  P1       \\
        \hline
        flag=1 & &    Y=1 & &   X=2 & &      X=2 &               \\
               &  r3=0 &   flag=1 & &  flag=1 & &      Y=1 &               \\
               &  r4=0 & &   r3=0 & &     r3=2 &     flag=1 &               \\
               &   Z=4 & &   r4=1 & &     r4=0 & &      r3=2     \\
               & flag=0 & &    Z=4 & &      Z=4 & &      r4=1     \\
        X=2 & & &  flag=0 & &    flag=0 & &       Z=4     \\
        Y=1 & &    X=2 & &   Y=1 & & &     flag=0    \\
        r1=2 & &    r1=2 & &   r1=2 & &      r1=2 &               \\
        r2=4 & &    r2=4 & &   r2=4 & &      r2=4 &               \\
        \hline
        \multicolumn{2}{|c||}{(r1,r2,r3,r4)=(2,4,0,0)} &   \multicolumn{2}{|c||}{(r1,r2,r3,r4)=(2,4,0,1)} &
        \multicolumn{2}{|c||}{(r1,r2,r3,r4)=(2,4,2,0)} &   \multicolumn{2}{|c|}{(r1,r2,r3,r4)=(2,4,2,1)} \\
        \hline
    \end{tabular}
\end{table}

Ο μοναδικός συνδιασμός για να μπορέσει ο Processor 0 να βγει από το inf loop, είναι ο Processor 1 να εκτελέσει την
flag=0.

\begin{table}[H]
    \centering
    \begin{tabular}{|c|c|}
        \hline
        P0     &     P1     \\
        \hline
        \hline
        X=2           &     \\
  Y=1           &           \\
  flag=1       &            \\
               &    r3=2    \\
               &    r4=1    \\
               &     Z=4    \\
               &   flag=0   \\
 r1=2          &            \\
 r2=4          &            \\
        \hline
    \multicolumn{2}{|c||}{(r1,r2,r3, r4)=(2,4,2,1)}  \\
        \hline
    \end{tabular}
\end{table}

\pagebreak

Για να επιτύχουμε το ίδιο αποτέλεσμα με weak ordering model πρέπει οι εντολές X=2,Y=1 να πραγματοποιηθούν πριν τις r3 =
X, r4 = Y. Συνεπώς θα πρέπει να έχουμε δύο SYNCH.
\begin{table}[H]
    \centering
    \begin{tabular}{|c|c|}
        \hline
        P0     &     P1     \\
        \hline
        \hline
        X=2           & while(flag==0);     \\
        Y=1           &         r3=X  \\
        SYNCH flag=1       &    r4=Y        \\
        while(flag==1)       &    Z=4    \\
        r1=X           &  SYNCH flag=0    \\
        r2=Z           &     \\
        \hline
    \end{tabular}
\end{table}

\end{document}

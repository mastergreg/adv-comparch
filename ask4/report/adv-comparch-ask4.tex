%{{{ preamble
\documentclass[a4paper,12pt]{article}
\usepackage{anysize}
\marginsize{2cm}{2cm}{1cm}{1cm}
%0.7\textwidth 6.0in \textheight = 664pt
\usepackage{xltxtra}
\usepackage{xunicode}
\usepackage{graphicx}
\usepackage{color}
\usepackage{xgreek}
\usepackage{fancyvrb}
\usepackage{minted}
\usepackage{listings}
\usepackage{enumitem} 
\usepackage{framed} 
\usepackage{relsize}
\usepackage{float} 
\usepackage{pstricks}
\usepackage{pst-node}
\usepackage{pst-blur}
\setmainfont[Mapping=tex-text]{FreeSerif}
%}}}
\begin{document}

\def\thesection {\arabic{section}. }
\def\thesubsection {\Roman{subsection}) }

\include{title/title}

\section*{Εισαγωγή}

%{{{ Προσομοίωση

\subsection{Προσομοίωση}
Ο κώδικας που μας δόθηκε ήταν ο ακόλουθος:

\inputminted[linenos,fontsize=\scriptsize,frame=leftline]{c}{files/partA1-default-partA.c}

%}}}

\pagebreak

%{{{ Ιεραρχία μνήμης και μοντέλο απόδοσης

\subsection{Ιεραρχία μνήμης και μοντέλο απόδοσης}

Χρησιμοποιήσαμε την ιεραρχία μνήμης όπως μας δίνεται στο Παράρτημα. Αυτή έχει
δύο επίπεδα κρυφής μνήμης L1 (2 way set associative 64 bytes $\times$ 512 lines
write-through LRU policy) και
L2 (4 way set associative 128 bytes $\times$ 1024 lines write-back LRU policy).

\begin{table}[H]
    \centering
    \begin{tabular}{| c | c | c | c | c |}
        \hline
           & assoc  &   line size &   lines & size\\
        \hline
        \hline
        L1 instruction cache   & 2  &   64  &   512 & 32768 = 32 KB\\
        \hline
        L1 data cache   & 2  &   64  &   512 & 32768 = 32 KB\\
        \hline
        L2 cache  & 4  &   128  &   1024 & 131072 = 128 KB\\
        \hline
    \end{tabular}
    \caption{Cache Hierarchy}
    \label{fig:T2}
\end{table}

Στο μοντέλο που προσομοιώνουμε οι κύκλοι υπολογίζονται χωρίς penalty από το
Simics. Για να υπολογίσουμε τους κύκλους θεωρούμε πως πρόσβαση στην L1 cache
γίνεται σε 1 κύκλο, στην L2 σε 20 κύκλους και στη μνήμη σε 300 κύκλους οπότε
έχουμε τον τύπο:

\begin{equation}
    Cycles = Inst + L1_{Accesses} * L1_{Time} + L2_{Accesses} * L2_{Time} +
    Mem_{Accesses} * Mem_{Time}
\end{equation} 

\begin{table}[H]
    \centering
    \begin{tabular}{| c | c |}
        \hline
        Cycles & 1523400216 \\
        \hline
        L1 miss ratio & 0.029 \\
        \hline
        L2 miss ratio & 0.015 \\
        \hline
    \end{tabular}
    \caption{Μετρικές πρώτης εκτέλεσης}
    \label{fig:T3}
\end{table}

%}}}

\section*{Τεχνικές Βελτιστοποίησης}
%{{{ Loop Interchange

\subsection{Loop Interchange}

Για την προσομοίωση κάναμε 6 εκτελέσεις για κάθε διάταξη των i,j,k όπως
φαίνεται στον πίνακα \ref{fig:T1}.

\begin{table}[H]
    \centering
    \begin{tabular}{| c | c |}
        \hline
        Σειρά προσομοίωσης & Διάταξη \\
        \hline
        \hline
        1   &  i,j,k \\
        \hline
        2   &  i,k,j \\
        \hline
        3   &  j,i,k \\
        \hline
        4   &  j,k,i \\
        \hline
        5   &  k,i,j \\
        \hline
        6   &  k,j,i \\
        \hline
    \end{tabular}
    \caption{Δυνατές Διατάξεις}
    \label{fig:T1}
\end{table}


Τα αποτελέσματα φαίνονται στα ακόλουθα σχήματα.

\begin{figure}[H]
	\centering
    \includegraphics[width=0.8\textwidth]{files/partA1-simics-outputs-cycles.png}
	\caption{Cycles}
	\label{fig:A1}
\end{figure}

\begin{figure}[H]
	\centering
    \includegraphics[width=0.8\textwidth]{files/partA1-simics-outputs-l1_miss.png}
    \caption{L1 Miss Rate}
	\label{fig:A2}
\end{figure}

\begin{figure}[H]
	\centering
    \includegraphics[width=0.8\textwidth]{files/partA1-simics-outputs-l2_miss.png}
    \caption{L2 Miss Rate}
	\label{fig:A3}
\end{figure}

\begin{table}[H]
    \centering
    \begin{tabular}{| c | c | c | c |}
        \hline
        Cycles & L1 Miss rate & L2 Miss rate & Speedup  \\
        \hline
        \hline
        1523400216 & 0.0285795820412 & 0.0150389941603   &  1.0            \\
        1519770751 & 0.00468022311872 & 0.0152373522682  &  0.997617523641 \\
        1145990549 & 0.0303714959475 & 0.0281574903722   &  0.752258360583 \\
        2752007165 & 0.0835636317199 & 0.06750286478     &  1.80648994013  \\
        1905041698 & 0.00455744098555 & 0.0115127159615  &  1.25051951417  \\
        2659818958 & 0.0833196681295 & 0.0619748730144   &  1.74597517452  \\
    \end{tabular}
    \caption{Μετρικές όλων των εκτελέσεων}
    \label{fig:T1}
\end{table}


%}}}

\pagebreak

%{{{ Cache Blocking

\subsection{Cache Blocking}

Η L1 cache έχει μέγεθος 32KB, και line size 64 bytes. Κάνουμε πράξεις μεταξύ
αριθμών κινητής υποδιαστολής σε 32bit σύστημα συνεπώς κάθε ένας έχει μέγεθος
4bytes. Κάθε cache line χωράει 16 αριθμούς συνεπώς για να εκμεταλλευτούμε
καλύτερα την τοπικότητα των αναφορών θα εκτελούμε κάθε loop σε 16δες. Έτσι θα
χρησιμοποιούμε δεδομένα που έχουν ήδη έρθει στην cache.

\inputminted[linenos,fontsize=\scriptsize,frame=leftline]{c}{files/partA2-v3_patched-partA.c}


\begin{table}[H]
    \centering
    \begin{tabular}{| c | c |}
        \hline
        Cycles & 1448776923 \\
        Cycles & 9385880    \\
        \hline
        L1 miss ratio & 0.0131 \\
        L1 miss ratio & 0.0026 \\
        \hline
        L2 miss ratio & 0.0292 \\
        L2 miss ratio & 0.0065 \\
        \hline
    \end{tabular}
    \caption{Διαφορά των εκτελέσεων}
    \label{fig:T4}
\end{table}

\begin{figure}[H]
	\centering
    \includegraphics[width=0.8\textwidth]{files/partA2-simics-outputs-cycles.png}
	\caption{Cycles}
	\label{fig:B1}
\end{figure}

\begin{figure}[H]
	\centering
    \includegraphics[width=0.8\textwidth]{files/partA2-simics-outputs-l1_miss.png}
    \caption{L1 Miss Rate}
	\label{fig:B2}
\end{figure}

\begin{figure}[H]
	\centering
    \includegraphics[width=0.8\textwidth]{files/partA2-simics-outputs-l2_miss.png}
    \caption{L3 Miss Rate}
	\label{fig:B3}
\end{figure}



%}}}

\section*{Cache Consistency}
\end{document}

%{{{ preamble
\documentclass[a4paper,12pt]{article}
\usepackage{anysize}
\marginsize{2cm}{2cm}{1cm}{1cm}
%0.7\textwidth 6.0in \textheight = 664pt
\usepackage{xltxtra}
\usepackage{xunicode}
\usepackage{graphicx}
\usepackage{color}
\usepackage{xgreek}
\usepackage{fancyvrb}
\usepackage{minted}
\usepackage{listings}
\usepackage{enumitem} 
\usepackage{framed} 
\usepackage{relsize}
\usepackage{float} 
\usepackage{pstricks}
\usepackage{pst-node}
\usepackage{pst-blur}
\setmainfont[Mapping=tex-text]{FreeSerif}
%}}}
\begin{document}

\def\thesection {\roman{section}: }

\include{title/title}

\section{Εισαγωγή}
Σκοπός της άσκησης αυτής ήταν η χρήση και εξοικείωση με το λογισμικό Simics της
Wind River, σε συνδυασμό με τη μελέτη της επίδρασης διαφόρων παραμέτρων των
μνημών L1 και L2 cache στην απόδοση συγκεκριμένων benchmarks.

Πιο συγκεκριμένα, χρησιμοποιήσαμε ένα εικονικό μηχάνημα που έτρεχε Fedora 5
εξετάσαμε τις μεταβολές στην επίδοση art, gzip, equake και mcf,
χρησιμοποιώντας ως μετρική απόδοσης το IPC (Instructions Per Cycle) και Miss
Rate.

%{{{ L1 cache, constant line size

\section{L1 cache, constant line size}

Στο τμήμα αυτό χρησιμοποιήσαμε διαφορετικές L1 cache κρατώντας σταθερό το
cache line size και τροποποιώντας το μέγεθος της cache και το associativity
όπως φαίνεται στον παρακάτω πίνακα. Ακόμη κρατήσαμε σταθερή την L2 cache με
χαρακτηριστικά: μέγεθος 1ΜΒ, line size 128B και associativity 8. Όλες οι cache
που χρησιμοποιήσαμε ακολουθούν πολιτική αντικατάστασης LRU (least recently
used).

\begin{table}[H]
    \centering
    \begin{tabular}{c c c c}
        Size & line size & lines & associativity\\ 
        \hline
        \hline
        16K   & 64 &  256  & 2\\
        16K   & 64 &  256  & 4\\
        \hline
        32K   & 64 &  512  & 4\\
        32K   & 64 &  512  & 8\\
        \hline
        64K   & 64 &  1024 & 4\\
        64K   & 64 &  1024 & 8\\
    \end{tabular}
\end{table}

Στα παρακάτω διαγράμματα παρατηρούμε πως αυξάνοντας το μέγεθος της cache το
IPC αυξάνει ελαφρώς. Ακόμη παρατηρούμε πως το benchmark art έχει πολύ μικρό
IPC πράγμα που δηλώνει πως δε χαρακτηρίζεται από τοπικότητα αναφορών και έτσι
δεν αξιοποιεί το μέγεθος της cache. Ωστόσο το miss-rate πέφτει πολύ
περισσότερο όσο το μέγεθος της L1 cache αυξάνει για κάθε benchmark.


\begin{figure}[H]
    \centering
    \includegraphics[width=0.8\textwidth]{files/part1a-ipc_all.png}
    \caption{IPC: L1 cache line size 64}
\end{figure}


\begin{figure}[H]
    \centering
    \includegraphics[width=0.8\textwidth]{files/part1a-missrate_all.png}
    \caption{miss-rate: L1 cache line size 64}
\end{figure}

%}}}

\pagebreak

%{{{ L1 cache, variable line size

\section{L1 cache, variable line size}

Στο τμήμα αυτό χρησιμοποιήσαμε διαφορετικές L1 cache μεταβάλλοντας αυτή τη
φορά το το cache line size και το μέγεθος της cache και το associativity όπως
φαίνεται στον παρακάτω πίνακα. Ακόμη κρατήσαμε σταθερή την L2 cache με
χαρακτηριστικά: μέγεθος 1ΜΒ, line size 128B και associativity 8. Όλες οι cache
που χρησιμοποιήσαμε ακολουθούν πολιτική αντικατάστασης LRU (least recently
used).

\begin{table}[H]
    \centering
    \begin{tabular}{c c c c}
        Size & line size &lines & associativity\\ 
        \hline
        \hline
        16K & 32    & 512   & 4\\
        16K & 128   & 128   & 4\\
        \hline
        32K & 32    & 1024  & 4\\
        32K & 128   & 256   & 4\\
        32K & 32    & 1024  & 8\\
        32K & 128   & 256   & 8\\
        \hline
        64K & 32    & 2048  & 4\\
        64K & 128   & 512   & 4\\
        64K & 32    & 2048  & 8\\
        64K & 128   & 512   & 8\\
    \end{tabular}
\end{table}

Τα αποτελέσματα που ακολουθούν είναι ξεχωριστά ανά cache line size ώστε να
ξεχωρίζει η επίδραση της κάθε μετρικής σε αυτά. Παρατηρήσαμε πως το μέγεθος
της cache είναι αυτό που παίζει το σημαντικότερο ρόλο στην επίδοση των
benchmarks, ενώ αξιοσημείωτο είναι το γεγονός πως το art δεν δείχνει να
επηρεάζεται από το μέγεθος της L1 cache, πράγμα που σημαίνει πως η υπόθεση
που κάναμε παραπάνω για την κακή αξιοποίηση της τοπικότητας φαίνεται σωστή.

\begin{figure}[H]
    \centering
    \includegraphics[width=0.8\textwidth]{files/part1b-32-ipc_all.png}
    \caption{IPC: L1 cache line size 32}
\end{figure}

\begin{figure}[H]
    \centering
    \includegraphics[width=0.8\textwidth]{files/part1b-32-missrate_all.png}
    \caption{miss-rate: L1 cache line size 32}
\end{figure}

\begin{figure}[H]
    \centering
    \includegraphics[width=0.8\textwidth]{files/part1b-128-ipc_all.png}
    \caption{IPC: L1 cache line size 128}
\end{figure}

\begin{figure}[H]
    \centering
    \includegraphics[width=0.8\textwidth]{files/part1b-128-missrate_all.png}
    \caption{miss-rate: L1 cache line size 128}
\end{figure}

%}}}

\pagebreak

%{{{ L2 cache, constant line size

\section{L2 cache, constant line size}
Στο τμήμα αυτό της άσκησης μελετήσαμε την επίδραση της μεταβολής του μεγέθους
και του associativity της L2 cache στην εκτέλεση των benchmarks. Τα
χαρακτηριστικά που θέσαμε για την L2 φαίνονται στον ακόλουθο πίνακα. Ακόμη
η L1 cache είχε μέγεθος 32K και cache line size 64B.

\begin{table}[H]
    \centering
    \begin{tabular}{c c c c}
        Size & line size & lines & associativity\\ 
        \hline
        \hline
        256K   & 128 & 2048  & 4\\
        \hline
        512K   & 128 & 4096  & 4\\
        512K   & 128 & 4096  & 8\\
        \hline
        1024K  & 128 & 8192  & 8\\
        1024K  & 128 & 8192  & 16\\
        \hline
        2048K  & 128 & 16384 & 8\\
        2048K  & 128 & 16384 & 16\\
    \end{tabular}
\end{table}

Καθώς μεταβάλλουμε το μέγεθος της L2 cache διαπιστώσαμε σημαντική βελτίωση στο
IPC και πτώση στο miss-rate κατά μία τάξη μεγέθους. Επίσης εδώ φαίνεται να
έχει σημασία και το μέγεθος του cache-line όπως δείχνουν τα αποτελέσματα
ειδικά στις δύο τελευταίες εκτελέσεις των benchmarks.

\begin{figure}[H]
    \centering
    \includegraphics[width=0.8\textwidth]{files/part2a-ipc_all.png}
    \caption{IPC: L2 cache line size 128}
\end{figure}

\begin{figure}[H]
    \centering
    \includegraphics[width=0.8\textwidth]{files/part2a-missrate_all.png}
    \caption{miss-rate: L2 cache line size 128}
\end{figure}

%}}}

\pagebreak

%{{{ L2 cache, variable line size


\section{L2 cache, variable line size}

Τέλος μεταβάλλουμε το το μέγεθος του cache-line και το μέγεθος της L2 cache
σύμφωνα με τον παρακάτω πίνακα κρατώντας τα χαρακτηριστικά της L1 όπως ίδια με
τις προηγούμενες εκτελέσεις.

\begin{table}[H]
    \centering
    \begin{tabular}{c c c c}
        Size & line size & lines & associativity\\ 
        \hline
        \hline
        512K   & 64  & 8192  & 8\\
        512K   & 256 & 2048  & 8\\
        \hline
        1024K  & 64  & 16384 & 8\\
        1024K  & 256 & 4096  & 8\\
        \hline
        2048K  & 64  & 32768 & 8\\
        2048K  & 256 & 8192  & 8\\
    \end{tabular}
\end{table}

Τα αποτελέσματα ακολουθούν χωρισμένα ανά μέγεθος cache-line. Παρατηρούμε πως
το IPC βελτιώνεται κατά πολύ όσο μεγαλώνει τόσο το μέγεθος της cache όσο και
του cache-line. Τα equake και gzip δείχνουν να μην επηρεάζονται από τις
αλλαγές στην L2 cache πράγμα που δηλώνει πως έχουν μικρό αριθμό προσβάσεων
στην L2.


\begin{figure}[H]
    \centering
    \includegraphics[width=0.8\textwidth]{files/part2b-64-ipc_all.png}
    \caption{IPC: L2 cache line size 64}
\end{figure}

\begin{figure}[H]
    \centering
    \includegraphics[width=0.8\textwidth]{files/part2b-64-missrate_all.png}
    \caption{miss-rate: L2 cache line size 64}
\end{figure}

\begin{figure}[H]
    \centering
    \includegraphics[width=0.8\textwidth]{files/part2b-256-ipc_all.png}
    \caption{IPC: L2 cache line size 256}
\end{figure}

\begin{figure}[H]
    \centering
    \includegraphics[width=0.8\textwidth]{files/part2b-256-missrate_all.png}
    \caption{miss-rate: L2 cache line size 256}
\end{figure}

\section{Συμπεράσματα}
Κυριότερο χαρακτηριστικό που επηρέασε την απόδοση της εκάστοτε cache ήταν το
μέγεθος της. Το associativity δεν έδειξε να επηρεάζει σχεδόν καθόλου ενώ το
μέγεθος του cache-line συνεισφέρει αλλά σε λιγότερο βαθμό εν γένει.
Επιπρόσθετα η ιεραρχία της cache δεν αποτελεί πανάκεια στην απόδοση των
προγραμμάτων που γράφει κάποιος. Είναι δυνατό η ίδια ιεραρχία μνήμης να
συμπεριφέρεται πολύ χειρότερα από κάποια φαινομενικά καλύτερη της σε ειδικές
περιπτώσεις.

%}}}


\end{document}
